% Chapter 1

\chapter{Introduction} % Main chapter title

\label{intro} % For referencing the chapter elsewhere, use \ref{Chapter1}

% \lhead{Chapter 1. \emph{Introduction}} % This is for the header on each page - perhaps a shortened title

In several activities, the intrinsic motivation is limited by the context of use. That is, once a given benefit or reward is obtained, people are likely to stop doing the activities. In educational environments, there exists a multitude of digital tools that facilitate the learning process. However, the lack of additional extrinsic motivational elements can make people stop using them. The present work provides a solution aimed at increasing user engagement in an educational tool. The solution is based on the design of a gamification strategy that includes a casual game. The results showed that there was no statistically significant variation in the user engagement between the proposed and a traditional gamification strategy.

%----------------------------------------------------------------------------------------
\section{Motivation}
Activities such as using a tool, executing an action, or adopting a behaviour have an intrinsic incentive. In many cases, such incentive is a reward in the form of a monetary compensation for a job or good marks in a study program. The lack of additional motivational factors can make people stop doing such activities once the reward is obtained. Likewise, activities that do not offer specific rewards but provide other benefits might also be affected by the absence of extrinsic motivational elements. In contrast, activities that people do for joy or entertainment are likely to be repeated over time.

In the specific case of spaced repetition tools, the main incentive to use them is the memorisation of new content. However, their flexibility to let users define the duration of each session and the interval between consecutive revisions can lead to a gradual reduction of its usage over time. Such circumstance prevents learners from keeping getting the benefits of the technique. These conditions set a perfect environment for the adoption of a gamification scheme that provides additional incentives to learners. This way, users of spaced repetition tools may mantain a constant pace of study while enjoying the experience.

%---------------------------------------------------------------------------------------
\section{Objective}
Gamification provides several elements that have been leveraged to improve user experience and increase user engagement. For their part, casual games have characteristics that have converted them into subjects of study beyond the leisure context. They have been proved to provide benefits in different areas including health, work, and study. Spaced repetition has been implemented in a multitude of pieces of software including mobile applications. Specifically, AnkiDroid \citep{raoul2012ankidroid} has become a popular application that implements a general solution for spaced repetition. Despite its popularity, the application lacks extrinsic motivational elements.

The objective of the present work is to design and implement a gamification strategy aimed at increasing the user engagement in AnkiDroid. Unlike traditional strategies, the proposed solution includes a casual game as an additional motivational element. The game is integrated such that the gameplay is connected with the revision of flashcards. This way, users are required to review flashcards to get additional benefits in the game.
The effectiveness of the proposed solution is then evaluated by analysing the results obtained from the data collected from participants.


%---------------------------------------------------------------------------------------
\section{Achieved Results}
The process started with the selection of components of interest in AnkiDroid. Those components were chosen based on their relevance from the users' perspective, and their suitability to be modified. Then, a set of game elements and schemes were added such that they were connected to one another to provide an integral gamification experience. After that, a casual game was included to provide another extrinsic motivational factor. This integration was done by connecting the game with the revision of flashcards using the previously defined gamification components as a link.

Then, the modified application was tested by two groups of participants. Each group of participants tested a different version of the application. One version contained all the additional elements, and the other did not have the casual game. The objective of the study was to collect data from the application. These data were then processed and analysed to evaluate the effectiveness of the inclusion of a casual game as part of a gamification strategy.

The analysis was done in terms of the variation of user engagement in both groups of participants. To do so, a set of user engagement metrics was defined based on the available information and its relevance. The results showed that there was no statistically significant difference in any of the metrics. Therefore, the inclusion of a casual game as part of a gamification strategy did not have a positive or negative impact on the user engagement. These results served as the basis to discuss potential issues in the design and propose further work.