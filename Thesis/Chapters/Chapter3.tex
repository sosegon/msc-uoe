% Chapter 1

\chapter{Related Work} % Main chapter title

\label{rela} % For referencing the chapter elsewhere, use \ref{Chapter1}

\lhead{Chapter 3. \emph{Related Work}} % This is for the header on each page - perhaps a shortened title

%----------------------------------------------------------------------------------------
\section{Gamification for learning}
The intrinsic characteristics of Gamification have gotten special attention in educational environments. Its ability to increase and mantain the motivation makes it a perfect fit to be integrated in educational tools aimed to improve the learning experience. The amount of mental effort required in the learning process greatly depends on the learner's perception about the source of knowledge \citep{salomon1983differential}. Therefore, when people perceive the source of knowledge in a possitive fashion, the experience is more pleasant and the results are better. In this context, there have been several approaches that have integrated gamification techniques in educational tools effectively.

Gamification is flexible to be used by different target audiences effectively. The work by \citep{boticki2015usage} used badges as motivational elements in a mobile learning system to explore the individual and collaborative learning of primary school pupils. Their results showed that the quality and quantity of contributions were correlated with the end-year assessment score. Likewise, \citep{slish2015gamification} showed how the use of gamification techniques increased the performance of undergraduate students, being lower performing students the ones who got the highest benefits. Finally, tools for the general public have also implemented gaming elements to make them more engaging \citep{morrison2014khan}.

Adaptation to different environments is another characteristic that makes Gamification an option to consider in learning tools. The work by \citep{su2015mobile} took advantage of the ubiquity nature of smartphones to improve outdoor learning activities using a MGLS (Mobile Gamification Learning System). Their results showed a positive relationship between the motivation and learning achievement. Web platforms for MOOC (Massive Online Open Courses) have found in Gamification a means to increase students' motivation and reduce dropout rates \citep{gene2014gamification}. Finally, gamification techniques have also been used in desktop environments to make tasks like surveys more enjoyable\citep{cheong2013quick}.

The adoption of gamification elements and techniques is facilitated due to its intrinsic characteristics of flexibility and adaptation. Depending on the context of usage, the target audience, and the platform of deployment of an educational tool, the inclusion of gamification elements and techniques can vary and have different goals. Nonetheless, the ultimate objective of gamifying an educational tool is to improve the learner's experience by providing further motivational elements to make it more enjoyable. Previous work has demonstrated that the integration of Gamification has been effective in different contexts and environments.

%----------------------------------------------------------------------------------------
\section{Previous attempts to gamify AnkiDroid}
