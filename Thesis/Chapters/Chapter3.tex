% Chapter 1

\chapter{Related Work} % Main chapter title

\label{rela} % For referencing the chapter elsewhere, use \ref{Chapter1}

\lhead{Chapter 3. \emph{Related Work}} % This is for the header on each page - perhaps a shortened title

Gamification has been considered as an alternative to improve user experience in educational contexts. Its characteristics of flexibility and adaptation have been leveraged to adopt educational tools from different platforms and targeted to distinct audiences. Anki has been a subject of several attempts to be gamified. Some of them were simply aimed at improving the visual aspect of the desktop interfaces, whereas others tried to provide a more complete gamification environment. However, the literature did not report any previous work related to gamifying the mobile interface of Anki (AnkiDroid).

%----------------------------------------------------------------------------------------
\section{Gamification for learning}
The characteristics of gamification have gotten special attention in educational environments. Its ability to increase and maintain the motivation makes it a perfect fit to be integrated into educational tools aimed at improving the learning experience. The amount of mental effort required in the learning process greatly depends on the learner's perception of the source of knowledge \citep{salomon1983differential}. Therefore, when people perceive the source of knowledge in a positive fashion, the experience is more pleasant and the results are better. In this context, there have been several approaches that have integrated gamification techniques in educational tools effectively.

Gamification is flexible to be used by different target audiences effectively. The work by \citep{boticki2015usage} used badges as motivational elements in a mobile learning system to explore the individual and collaborative learning of primary school pupils. Their results showed that the quality and quantity of contributions were correlated with the end-year assessment score. Likewise, \citep{slish2015gamification} showed how the use of gamification techniques increased the performance of undergraduate students, with lower-performing students getting the highest benefits. Finally, tools for the general public have also implemented gaming elements to make them more engaging \citep{morrison2014khan}.

Adaptation to different environments is another characteristic that makes gamification an option to consider in learning tools. The work by \citep{su2015mobile} took advantage of the uniquitous nature of smartphones to improve outdoor learning activities using a mobile gamification learning system (MGLS). Their results showed a positive relationship between the motivation and learning achievement. Web platforms for massive online open courses (MOOCs) have found in gamification a means to increase students' motivation and reduce dropout rates \citep{gene2014gamification}. Finally, gamification techniques have also been used in desktop environments to make tasks like surveys more enjoyable \citep{cheong2013quick}.

The adoption of gamification elements and techniques is facilitated due to its characteristics of flexibility and adaptation. Depending on the context of usage, the target audience, and the platform of deployment of an educational tool, the inclusion of gamification elements and techniques can vary and have different goals. Nonetheless, the ultimate objective of gamifying an educational tool is to improve the learner's experience by providing further motivational elements to make it more enjoyable. Previous work has demonstrated that the integration of gamification has been effective in different contexts and environments.

%----------------------------------------------------------------------------------------
\section{Previous attempts to gamify Anki}
The desktop version of Anki has the ability to increase the available features by means of add-ons. Taking advantage of this characteristic, there have been several attempts to gamify the tool. The majority of the add-ons are aimed at improving the visual aspect of Anki by implementing game elements like progress bar \citep{glut2017progress} or random rewards \citep{glut2017puppy}. The mechanics of these elements are straightforward, and their main objective is to provide visual feedback to the user. Even though these options make the user interface more appealing, they lack game schemes to encourage the user to keep using the tool.

Other options include more advanced mechanics to integrate game schemes in Anki. The first attempt aimed at giving Anki a more game-like environment was AnkiWarrior \citep{proxx2010warrior}. The mechanics of the game were similar to a role-playing game (RPG) where the learner played the role of a warrior conquering Japanese cities. In this role, the player had to convince the inhabitants of a city to join the empire. Then, the player had to review more flashcards to convince more inhabitants until the entire city was part of the empire. This alternative made Anki more appealing, but it only implemented a scheme that mapped the number of reviewed cards with the number of citizens convinced by the warrior.

Following the path started by AnkiWarrior, AnkiEmperor \citep{proxx2012emperor} added more game elements and schemes to Anki. Similarly, the learner played a role, but in this case, the objective was to become an emperor. To achieve that objective, the learner had to obtain gold to build constructions like parks, skyscrapers, and landmarks. The gold was also used to unlock new cities. Ultimately, the rank of a player improved based on the amount of gold and the number of constructions. Eventually, the rank was sufficiently high to become an emperor. The learner was motivated to review flashcards in order to earn gold, which was the main resource in the game.

These previous attempts to gamify Anki demonstrated the interest in making the software more appealing for the users. The majority of them focused only on improving the visual appearance of Anki, while others implemented actual game elements and schemes to it. However, the main drawback of these alternatives was their narrow availability since they have been solutions for the desktop interface only (registers of gamified versions of AnkiDroid were not found). In addition, the lack of competitive elements, an essential component of game schemes, might have reduced the overall benefits of these solutions. Nonetheless, this work can be considered for further attempts to gamify Anki.
