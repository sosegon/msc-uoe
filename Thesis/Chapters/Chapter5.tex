% Chapter 5

\chapter{Experimental setup} % Main chapter title

\label{expe} % For referencing the chapter elsewhere, use \ref{Chapter5}

\lhead{Chapter 5. \emph{Experimental setup}} % This is for the header on each page - perhaps a shortened title

The modified AnkiDroid application was set for an experimental stage. The main objectives of this stage were collecting data and getting feedback from people. This information was then used to evaluate the effectiveness of the gamification strategy. Therefore, it was necessary to set an scheme to recruit participants, distribute the application, and collect data.

%----------------------------------------------------------------------------------------
\section{Data collection}
There are two types of data that were be used to evaluate the application: quantitative and qualitative. Both types of information give different perspectives about the usage of the application. Moreover, when used together, they can complement their insights about the overall gamification strategy and its different elements. Given their differents characteristics, they schemes to collect them were different. First, quantitative data were indirectly generated by users through the usage of the application, so that, users did nothing else than using application. In the other hand, qualitative data required the direct participation of users to provide this type of information.

\subsection{Quantitative data}
Originally, AnkiDroid collects a lot of usage information which is presented to users as statistics. These data give users a perspective about their progress based on the number of flashcards review, the number of sessions, the amount of time in sessions and more. The application even forecasts the number of flashcard to be reviewed in the near future. Even though the availability of this type of information, it is not used to evaluate the strategy because it provides a general overiew of the usage from the perspective of users, and discards many details.

\subsubsection{Types of information}
The information collected in the original application also lacks details related to the integrated game and the gamification elements. Therefore, it was necessary to collect information relevant and specific to the implemented gamification strategy. This information is generated in points where the user interacts with the application like playing the game or reviewing flashcards. There, the users do specific actions that can provide clues about their behaviours and reason about them. The collected information can be grouped into four categories as seen in Table \ref{tab:info-type}.

\begin{table*}[!htb]
	\centering
	{\renewcommand{\arraystretch}{2}
		\begin{tabular}{|R{3cm}|R{6cm}|}
		\hline
		\multicolumn{1}{|>{\centering\arraybackslash}m{2cm}|}{\textbf{Type}} &
		\multicolumn{1}{>{\centering\arraybackslash}m{6cm}|}{\textbf{Description}} \\
		\hline
		Common & Details of time and user.\\
		\hline
		Game & Details about the game.\\
		\hline
		Gamification & Details of game elements. \\
		\hline
		Anki & Details of flashcards and decks. \\
		\hline
		\end{tabular}
	}
	\caption{Types of information collected from the application}
	\label{tab:info-type}
\end{table*}

The first type of information is categorized as common. These data are meant to identify the time, date and user that generate them; they are recorded in every relevant interaction of the user with the application. The next type of information is related to the game; it includes details about the score, cheat tricks, and state of the grid. Game information can provide insights about the effectivenes of the integration. The gamification category has to do with the added game elements to AnkiDroid, it includes details about the number of coins, number of points, and names of the ankimals. Finally, the anki category refers to details about the decks, flashcards, and revision times.

\subsubsection{Type of logs}
The previously described types of information were stored in the form of logs. The logs are highly linked to the actions done by the users. For this reason, the number of types of logs is bigger than the number of types of information. Nonetheless, the logs can be grouped into two main classes: game logs and anki logs. Games logs are those containing the game type of information, however, some data from the gamification type of information is registered in these logs. Anki logs contain primarily anki type of data with some data of the gamification type. Common type of information is recorded in both groups of logs.

Specifically speaking there are 6 types of game logs and 9 types of anki logs as shown in Table \ref{tab:log-types}. Game logs are expected to be generated in the game context. Something similar occurs with anki logs, however, the design of the solution allows one anki log to be generated in the game context: Check leaderboard. This is possible because user can open the leader-board either in the game context or in the anki context. The reason for this design is that the leaderboard provides information of social aspect. Despite the fact that the leader-board display gamification data (points), users are not asked to do an specific action in anki or the game contexts to see their positions.

\subsection{Qualitative data}
Collecting qualitative data required the direct participation of the users to complete a survey. The survey had a set of questions aimed to gather information about the perception of the participants about the application. The majority of the questions required the user to select one or more options, however, the participants were also provide additional comments or suggestions. Even though, the survey was the main source of qualitative data, additional comments and suggestions about the application were given by participants and other people via email or comments in the forums and communities where the application was advertised for the general public.

%----------------------------------------------------------------------------------------
\section{Distribution}
The application was distributed in the Google Play Store platform. There, participants could easily find and install the application in their devices. Another advantage of this type of platform is the different testing stages it offers. This feature was used to set a beta testing stage to find potential issues and make improvements to the application before making it available to participants. In addition, the platform allows to make updates to correct bugs or add new features; such updates are automatically installed in the devices that already has the application. This kind of feature was useful to make the gamification elements available at different points during the study period. Finally, Google Play Store provides insights about the usage of the application, and users have the option to provide additional qualitative data in the form of comments.

%----------------------------------------------------------------------------------------
\section{Participants}
The main requirements for the participants was to have an Android device. It was not necessary that participants have an specific profile or characteristics since the original AnkiDroid application is targeted for all audiences. Participants were recruited during the developing of the application. During this process they were told about the objective of the study and the duration. They also were instructed about how to use the application, specially the feature related to the gamification strategy. Even though they were told about the duration of the study, none of them was forced to be part of it during the entire period.

%----------------------------------------------------------------------------------------
\section{Experimental and control group}
As described in section \ref{desi-gamification-strategy}, the main design decision was the inclusion of a casual game to set the stage to add other gamification elemnts. Since the inclusion of this component was the core of the gamification strategy, the application was split in two versions. The first version has the causal game and all the other gamification elements; this version was named AnkiGame. The second version did not have the integrated game, but the other gamification elements; this version was named AnkiPlay. The objective of this decision was to stablish an scenario to analyse the effectiveness of the gamification strategy based on the integration of the casual game.

The two versions of the application suppoused that the participants were split in two groups to test each of the versions. Therefore, the control and experimental groups were set. Participants in the first group were given the AnkiPlay version, whereas participants in the second group were given the AnkiGame version. None of the participants in each group were told about the existence of another group of participants to reduce any potential bias in the usage of the application.

%----------------------------------------------------------------------------------------
\section{Duration of study}
The study was set to last for four weeks were the participants used the application and generated data. The study period was divide in two stages. The first stage lasted three weeks and the users were asked to use the application at least during this period. At the end of this period, participant could stop using the application. Once the main staged ended, it started the second stage when participants were asked to complete the survey, and they could still use the application if they wanted to. The objective of splitting the study in two phases was to have more insights about the usage of the application from those participants who decided to keep using it.

%----------------------------------------------------------------------------------------
\section{Broader audience}
Since the application was openly distributed in Google Play Store, it was possible to make it available to a broader audience. Therefore, it was possible to get more data from other people and have more insights about the effectiveness of the application. In this case, the application was advertised in AnkiDroid forums, therefore, it is likely the majority of new users had a previous contact with the original application. In this case, the information from these new users were isolated from the information from the original participants to avoid any incorrect analysis. The users that were not part of the original study were also informed about the nature and objectives of the application.