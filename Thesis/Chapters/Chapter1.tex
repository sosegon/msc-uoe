% Chapter 1

\chapter{Introduction} % Main chapter title

\label{intro} % For referencing the chapter elsewhere, use \ref{Chapter1} 

\lhead{Chapter 1. \emph{Introduction}} % This is for the header on each page - perhaps a shortened title

%----------------------------------------------------------------------------------------
\section{Gamification}
Gamification can be defined as the process of adding game elements and mechanics in non-game contexts \citep{deterding2011game}. The main objective of Gamification is to improve the user's experience and increase the motivation to use a product or service. To accomplish such objective, Gamification takes advantage of the inherent nature of humans to play. Unlike mandatory activities such study and work, play is voluntary and free; its main outcome is a feeling of joy and excitement \citep{johan1950homo}. These conditions set the environment for the adoption of game concepts and techniques in broader contexts.

Over the past ten years, Gamification has attracted the attention of industry and academia. In the industry, companies have found a means to improve the performance and commitment of employees by avoiding traditional schemes of monetary rewards and punishments. Morevover, Gamification provides a set of tools to increase the loyalty and engagement of users and customers. For the academia, Gamification has expanded and merged various field of research given its interdisciplinary nature. It has attracted the attention of researchers in areas such as Human Computer Interaction, Software Development, Psychology, Pedagogy, Bussiness Management and others.

%----------------------------------------------------------------------------------------
\section{Spaced Repetition}
Spaced Repetition is a technique that facilitates the retention of new knowledge. It leverages the spacing effect phenomenon to help learners memorize specific contents \citep{hintzman1974theoretical}. This phenomenon allows learners to increase their capacity of retention by acquiring new knowledge in short recurrent sessions rather than in a single massive revision. In its basic form, Spaced Repetition sets increasing intervals of time between subsequent review sessioins of previously learned material. This means that the more challenging the content, the more frequently is reviewed by a learner. Then, the frequency of repetition is adjusted as the learner progresses.

Among the various existing techniques for memorization, Spaced Repetition stands out due to its simplicity and flexibility. The duration of each revision along with the inteval time between consecutive sessions is defined by the learner. In addition, the learner assesses the easiness of the content under revision to determine the frequency of repetition. Finally, Spaced Repetition is a technique that can be used to learn new content from any field, but it is specially useful when the number of items to memorize is large. Such characteristics allow the implementation of this technique as a piece of software, making it available to a wider audience.

%----------------------------------------------------------------------------------------
\section{Motivation}
Activities such as using a tool, executing an action, or adopting a behaviour have an underlying incentive. In many cases, such incentive is the obtaining of a reward in the form of a monetary compensation for a job, or good marks in a study program. The lack of additional incentive factors can make people stop doing such activities once the reward is obtained. Likewise, activities that do not offer specific rewards but provide other benefits are also affected by the absence of extra incentive elements. In contrast, activities that people do for joy or entertainment are likely to be repeated over time. 

In the specific case of Spaced Repetition, the main incentive to use it is the memorization of new content. However, its flexibility to let users define the duration of each session and the interval between consecutive revisions can lead to a gradual reduction of its usage over time. Such circumstance avoids the learners to keep getting the benefits of the technique. These conditions set a perfect environment for the adoption of a gamification scheme that provides new incetives to the learners. This way users of Spaced Repetition can mantain a constant pace of study while enjoying the experience.
%---------------------------------------------------------------------------------------