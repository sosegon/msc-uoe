% Chapter 6

\chapter{Conclusions} % Main chapter title

\label{conc} % For referencing the chapter elsewhere, use \ref{Chapter1}

\lhead{Chapter 7. \emph{Conclusions}} % This is for the header on each page - perhaps a shortened title

The present work described the process of gamifying an educational tool along with an evaluation based on the results obtained from data collected from new users of the application. The discussion was based around potential issues in the designed gamification strategy as well as the profile of the selected participants. It is important to note that gamification and other techniques that include extrinsic motivational elements are unlikely to give good results if the users lack the intrinsic motivation. Finally, a set of future projects were proposed based on the results and potential issues of the presented solution.

%----------------------------------------------------------------------------------------
\section{Discussion}
Gamification provides a way to include extrinsic motivational elements in different contexts. In educational tools, these elements can be used to make the learning experience more appealing, which may be reflected in a higher user engagement. The current project presented a gamification strategy in AnkiDroid aimed at increasing user engagement. The key difference with other gamification alternatives was the integration of a casual game as an additional motivational element. The results showed that the proposed alternative did not have a statistically significant variation of user engagement compared to a traditional gamification strategy.

The structure of the solution was aimed at creating a link between the revision of flashcards and the casual game such that users had an extrinsic motivation to review more flashcards. However, one of the potential issues of the solutions might have been the flow of the connection between the game and the flashcards revision. Switching between the game and the revision of flashcards used the deck picker as an intermediary point. The design was done in that way considering that the deck picker was the place in the application to get access to other functionalities.

Since the participants selected for the study did not have previous experience using AnkiDroid, the learning curve to use the application might have been another potential issue. AnkiDroid is a mature application, but it contains many features that might be difficult to understand at the beginning. Moreover, the application presents information to users that makes sense as long as they understand some concepts of spaced repetition. In addition, the modifications done to the application added an extra complexity to the application even though they were implemented so as not to be intrusive.

In addition, it is important to note that gamification provides the tools to include additional extrinsic motivational elements. It is critical that the users have the intrinsic motivation, i.e. learning something. If users are not interested in getting the benefits of an educational tool, other extrinsic motivational elements are unlikely to help. In the study, the participants might have lacked the intrinsic motivations; hence, they might have been lacking interest in the application overall.

%----------------------------------------------------------------------------------------
\section{Future work}
The presented work used a gamification strategy that integrated a casual game aimed at increasing user engagement in AnkiDroid. The obtained results were non-conclusive about the gamification strategy, and further evaluation must be performed. However, this does not mean that gamification has to be discarded as an alternative to improve the user experience. Based on the potential issues discussed before, additional work can be done aimed at testing the effectiveness of the gamification strategy with people that are guaranteed to have the intrinsic motivation, e.g. students of different educational levels, employees starting a new job, or people interested in learning new topics from a specific field.

Gamification techniques can also be studied in the context of facilitating the use of tools to new users. As discussed, AnkiDroid has an important level of complexity that requires users to spend some time to understand its features and functionalities. A gamification strategy could be aimed at providing a better user experience for new users and increasing retention. Moreover, the effects of gamification can also be studied in more advanced users only.